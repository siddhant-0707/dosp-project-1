\documentclass{article}
\usepackage{amsmath}
\usepackage{hyperref}

\title{Project 1: Sums of Consecutive Squares}
\date{Distributed Operating Systems Principles - Fall 2025}

\begin{document}

\maketitle

\section{Problem Definition}
An interesting problem in arithmetic with deep implications to elliptic curve
theory is the problem of finding perfect squares that are sums of consecutive
squares. 
A classic example is the Pythagorean identity:

\begin{equation}
    3^2 + 4^2 = 5^2
\end{equation}

This reveals that the sum of squares of 3 \& 4 is itself a square. A more interesting example is Lucas' Square Pyramid:

\begin{equation}
    1^2 + 2^2 + ... + 24^2 = 70^2
\end{equation}

In both of these examples, sums of squares of consecutive integers form the
square of another integer.
The goal of this first project is to use \textbf{Gleam} and the actor model to build a
good solution to this problem that runs well on multi-core machines.

\section{Requirements}
\textbf{Input:} The input provided (as command line to your program, e.g.  \texttt{lukas})
will be two numbers: $N$ and $k$. 
The overall goal of your program is to find all $k$ consecutive numbers starting at 1 or higher, and up to $N$, such that the sum of squares is itself a perfect square (of an integer).

\hspace{1cm}

\noindent\textbf{Output:} Print, on independent lines, the first number in the sequence for each
solution.

\hspace{1cm}

\noindent\textbf{Example 1:}

\texttt{lukas 3 2}

\texttt{3}

\hspace{1cm}

\noindent indicates that sequences of length 2 with start point between 1 and 3 contain 3, 4 as a solution since \(3^2 + 4^2 = 5^2\).

\newpage

\noindent\textbf{Example 2:}

\texttt{lukas 40 24}

\texttt{1}

\hspace{1cm}

\noindent indicates that sequences of length 24 with start point between 1 and 40 contain 1, 2 , ..., 24 as a solution since \(1^2 + 2^2 + ... + 24^2 = 70^2\).

\subsection{Actor modeling} In this project you have to use exclusively the actor facility in \textbf{Gleam} \textit{(projects that do not use multiple actors or use any other formof parallelism will receive no credit).} 

A model similar to the one indicated in class for the problem of adding up a lot of numbers can be used here, in particular define worker actors that are given a range of problems to solve and a boss that keeps track of all the problems and perform the job assignment.

\subsection{README File} In the README file you have to include the following material:

\begin{itemize}
    \item The size of the work unit that you determined results in best performance for your implementation and an explanation on how you determined it. 
    \begin{itemize}
        \item A size of the work unit refers to the number of sub-problems that a worker gets in a single request from the boss.
    \end{itemize}

    \item The result of running your program for \texttt{lukas 1000000 4}
    
    \item The \textbf{REAL TIME} as well as the ratio of \textbf{CPU TIME} to \textbf{REAL TIME} for the above, i.e. for \texttt{lukas 1000000 4}.
    \begin{itemize}
        \item The ratio of \textbf{CPU TIME} to \textbf{REAL TIME} tells you how many cores were effectively used in the computation. 
        \item If your ratio is close to 1, you have almost no parallelism and points will be subtracted.
    \end{itemize}
    
    \item The largest problem you managed to solve.
\end{itemize}

\section{BONUS – 15\%}
Use remote actors and run your program on 2+ machines. 
Use your solution to solve a large instance such as: \texttt{lukas 100000000 20}.
\textbf{To receive bonus points you must record a video demo and explain your solution.}

\end{document}
